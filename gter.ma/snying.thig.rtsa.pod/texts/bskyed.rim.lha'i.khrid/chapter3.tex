\addtocontents{toc}{\protect\newpage}
\section{Чистый результат. \\ Достижение состояния единства}

\subsection{Сущность Плода: \\ Общее представление}

Практикуя таким образом, самадхи пустотности очищает чистую Мудрость, тогда как
визуализация божества вместе с самадхи подобным музыкальному сопровождению очищает
загрязнения, которые приводят к вовлечению в обычные состояния существования. Понимая,
что божество является естественным проявлением Мудрости, а не чем-то
воспринимающимся как внешний объект, привычные тенденции, связанные с
концептуальным мышлением, исчезают. Далее, трансформация основы всего (kun gzhi gnas
'gyur) приводит к Мудрости Дхармадхату. Также и в "Сутре вхождения в три Тела (sku gsum
la 'jug pa'o mdo)" сказано:

\begin{verse}
Сознание основа всего, \\
растворяясь в Пространстве, \\
образует Отражающую Мудрость. \\
Ментальное сознание, \\
растворяясь в Пространстве,\\
образует Мудрость равенства. \\
Загрязненный ум, \\
растворяясь в Пространстве, \\
образует Различающую Мудрость.\\
И пять сознаний органов чувств, \\
растворяясь в Пространстве, \\
образуют Всеосуществляющую Мудрость.
\end{verse}

В этот момент происходит множество изменений. На внешнем уровне проявления
трансформируются в Будда-сферы (zhing khams). Внутренне скандхи трансформируются в
форму божества, а на тайном уровне восемь собраний (сознаний) трансформируются в
Мудрость. Это то, что рассматривается как ставшее "Всемерно Просветленным (kun tu sangs
rgyas)". Как сказано в "Семидесяти строфах прибежища (skyabs 'gro bdun cu pa)":

\begin{verse}
Поскольку пробуждаются ото сна неведения\\
И расширяют ум, охватывающий все познаваемое,\\
Состояние Будды распускается, подобно лепесткам лотоса.
\end{verse}

В соответствие с подходом "Великой магической сети (sgyu 'phrul drwa ba chen po)",
Пять Тел одновременно достигают совершенства, кактолько актуализируется лотосовый
уровень не-привязанности (ma chags padma сап), а также уровень великого накопления
чакры слогов (yi ge 'khor lo chags chen). Здесь я привожу лишь краткий обзор этих пяти [Тел].\\

\subsubsection{Важдракая}

Неизменная Ваджракая описывается в "Сети Мудрости (ye shes drwa ba)":
Чистота Пространства - это Ваджракая,
Неизменная, неописуемая и немыслимая.
Единственный путь, используемый всеми Буддами, это путь Естественного сияния
(rang bzhin gyi 'od gsal), Изначальное Пространство (gdod ma'i dbyings). В Реальности (chos
nyid) кульминация (mthar thug) этого процесса неизменна, а ее ваджрная природа постоянна
и необусловлена, отсюда и термин "ваджракая". Исходя из рассмотрения того, что она
естественно чиста, изначально свободна от загрязнения, а также совершенно чиста, в том
смысле, что она свободна от всех форм случайного загрязнения (glo bur gyi dri та),
Ваджракая также называется "Состоянием Будды с двойной чистотой (dag ра gnyis Idan gyi
sangs rgyas)".

\subsubsection{Абхисамбодхикая}

Что касается Абхисамбодхикаи, то в том же тексте сказано:

\begin{verse}
Абхисамбодхикая описывается\\
Как чистая и как свободная от нечистоты,\\
Совершенная, поскольку качества развиты,\\
А также она единая, поскольку нераздельная.
\end{verse}

Когда сияющая ясность природы ума достигает своего предела (sems nyid 'od du gsal
ba mthar thug), то она становится частью этой двойной чистоты. Что касается
проявляющегося аспекта, то присутствуют все уникальные просветленные качества
Состояния Будды. Сюда входят десять сил (stobs), четыре бесстрашия (mi 'jigs ра),
восемнадцать отдельных качеств Состояния Будды (sangs rgyas kyi chos та 'dres pa), великое
сострадание (thugs rje chen ро), тридцать семь факторов Просветления (byang chub kyi chos) и
т.д. Короче, она охватывает все неописуемые качества знания, любви и способностей
(mkhyen brtse nus), именно поэтому используется название "абхисамбодхикая" (mngon par
rdzogs par bang chub pa'i sku). Это Тело является основой для возникновения всех
уникальных качеств Состояния Будды.

\subsubsection{Дхармакая}

Третье из Тел - это безмятежная Дхармакая (zhi ba chos kyi sku). Дхармакая не
является лишь пустотой, поскольку это не просто Мудрость осознавания (rig ра ye shes).
Когда ее рассматривают с точки зрения самой Реальности, то Дхармакая - это Ваджракая, как
было объяснено выше, тогда как с точки зрения многообразия ее проявлений и отдельной
сущности, это Абхисамбодхикая, которая также была описана. Так как же Дхармакая
определяется в данном контексте? Она не постоянна, поскольку она за пределами обозрения
и осмысления. Но она также не является отсутствием, поскольку она является Мудростью,
различающего самоосознавания (so so rang rig pa'i ye shes). Она не является тем и другим или
же ни тем ни другим, поскольку ни постоянство, ни полное отсутствие не устанавливаются.
Следовательно, она имеет природу Пространства без центра и краев, как небо. В этом
Пространстве очень тонкая Мудрость смешивается в одном вкусе (го gcig). И хотя это
похоже на новолуние, поскольку неочевидно (mi mngon ра), ее познающий аспект, в смысле
внутренней ясности Мудрости (ye shes kyi nang gsal), является непрерывным.\\
\\
Функции Дхармакаи, как сущность развертывающейся Мудрости (ye shes mched ра),
это ясность, которая направлена вовне от медитативного состояния. Как таковая, она
функционирует как причина для Рупакаи (gzugs sku), для воплощающихся форм, которые
проявляются для сыновей победоносного (rgyal sras), которые пребывают на разных уровнях
бодхисаттвы, а также для обычных существ. Сюда входят формы, которые проявляются для
их глаз, просветленная речь, которую они слышат, запах дисциплины благородных, вкус
Дхармы, блаженное чувство медитативного сосредоточения, а также знание, связанное с
концептуальным анализом, который оценивает явления (chos la 'jal ba). По этой причине она
рассматривается как "Мудрость, которая направлена вовнутрь и не является двойственной".
В тексте "Гуру магической сети (sgyu 'phrul Ыа та)" сказано:\\

\begin{verse}
Дхармакая - это безобъектная основа для возникновения;\\
Она является внутренне ясной и очень тонкой Мудростью.\\
Эти три Тела, внутренне ясное Пространство, очень трудно обрести.
\end{verse}

\subsubsection{Самбогакая}

Четвертым [Телом] является Самбхогакая (longs spyod). В "Этапах пути" сказано:

\begin{verse}
Самопроявление осознавания,\\
Спонтанно совершенное, возникает как масса лучей света,\\
Миры, небесные дворцы, троны и украшения.
\end{verse}

Как здесь отмечено, просветленные формы, наделенные пятью определенностями
(nges ра lnga), проявляются из внутренней ясности, из Пространства Реальности. Подобно
ясным проявлениям, которые появляются, когда лучи солнца проникают в кристалл, они
проявляются естественным образом, как воплощения (bdag nyid) океана знаков и примет.
Таковы проявления самого Состояния Будды, регентов, которые являются учителями пяти
семей Будды (rigs Inga’i ston ра). Они проявляются подобно пустым формам и радуются
постоянно вращающемуся колесу непрерывности (rgyun gyi 'khor lo). И поскольку они
особые (thun mong та yin ра), то даже те, кто достиг десятого уровня не видят их, поскольку
они еще не отбросили завесы (sgrib ра) полностью и им еще предстоит брести око разума
(bio mig), которое видит все разнообразие качеств, связанных с тем каким все является на
самом деле и как это проявляются. В "Высшем потоке (rgyud Ыа та)" сказано:

\begin{verse}
Не то, о чем можно сказать, относящееся к абсолютному.\\
Не объект концептуализации, вне любых примеров.\\
Поскольку нет ничего выше, не относится к существующему или покою.\\
И даже благородные не могут охватить сферу победоносных.
\end{verse}

Чудесный дворец и другие элементы, связанные с совершенным местом, проявляются
и возникают на основании этого базового ясного сияния. Это можно сравнить с чистыми
проявлениями, которые возникают во сне. Когда они проявляются для тех, кто устранил все
завесы, то они не воспринимаются как реальные и конкретные вещи, существующие в каком-
то другом месте. И это похоже на то, что различные пустые формы могут проявляться для
йогина, чьи энергии вошли в центральный канал. Однако другие, находящиеся в том же
месте все еще не могут видеть эти формы.

Когда свет кристалла втянут вовнутрь, то он остается в основе для проявления этого
спектра. И точно также три Тела этой внутренней ясности (nang gsal) по-отдельности
присутствуют внутри базового Пространства тонкой Мудрости. Следующий пример
помогает это понять. Когда солнце присутствует как условие, лучи света проецируются из
кристалла. И точно также как элементы, проявляющиеся для нас, возникают объективно,
внешнее качество Мудрости сияет как формы, наделенные всеми знаками и чертами
Состояния Будды.

\subsubsection{Нирманакая}

Пятым [Телом] является Нирманакая, которая может проявляться как все что угодно.
Здесь мы имеем то, что называется "тайными учителями, кормчими, которые направляют
сынов победоносных, благородных и других существ к острову безмятежности". Эти учителя
являются отражениями Самбхогакаи, которая проявляется в восприятии высших учеников.
И хотя они проявляются как тождественные великой Самбхогакае Пространства,
состоящей из проявляющихся для себя элементов, это не то, чем они являются на самом
деле. Если обратиться к примеру, то разница между этими двумя подобна разнице между
отражением в зеркале и реальной вещью. Подобно отражению, проявляющееся напоминает
Самбхогакаю всеми своими знаками и приметами. Однако, эти чистые миры, свита и другие
подобные элементы проявляются для других, а также находятся внутри проявлений десятого
уровня. По этой причине "Тантра солнца и луны (nyi zla kha sbyor)" классифицирует их как
"наполовину Нирманакая и наполовину Самбхогакая".

Поскольку они являются естественной эманацией того, что проявляется для самого
себя (rang snang gi rang bzhin 'phrul pa), они известны как "естественная Нирманакая чистых
сфер (zhing khams)". Эти сферы называются "Непревзойденной", "Совершенной радостью",
"Славной", "Блаженной" (еще известной как "Лотосовый холм"), а также "Осуществление
высшей активности". В этих пяти сферах Вайрочана и другие Будды пяти семей учат свое
чистое окружение учеников. На десятом уровне они учат их природе пяти семей Будд, пяти
Дхармам и пяти трансформациям, посвящая их великим светом ('odzer chenpos dbang bskur).
В "Иллюзии актуализации Просветления (sgyu 'phrul mngon byang)" сказано:

\begin{verse}
Достигнув чистых уровней,\\
А также из полного совершенства пяти учителей,\\
Пяти высших Учений и пяти Мудростей,\\
Переходят к сущности совершенного Просветления.
\end{verse}

Из этого состояния возникают как "Нирманакая, которая укрощает существ". Форма
Нирманакаи происходит из благих кармических семян (sa bon las) существ, и она
проявляется так же как луна отбрасывает свое отражение в воде. И точно так же как луна
имеет силу отбрасывать отражение, Самбхогакая имеет силу проецировать эманации,
которые проявляются в восприятии тех, кто нуждается в руководстве. И так же как вода
является причинным звеном, позволяющим отражению луны появляться, те, кто нуждаются
в руководстве, имеют заслугу (bsod nams), которая позволяет проявляться этим эманациям. И
когда эти две вещи сходятся вместе, проявляющиеся отражения возникают, чтобы укротить
существ необходимым способом, так же как отражение луны в воде появляется без усилий.\\
\\
В соответствие с кармой каждого индивидуума эти эманации могут принимать форму
высших сфер, которые проявляются вверху, животных, которые проявляются сбоку, или
существ ада и духов, которые проявляются внизу. Они работают ради блага шести классов
существ, которые испытывают все виды страдания, соответственно восприятию каждого
уровня существования. Ввиду этого они заботятся о благополучии других, воплощаясь как
существа осознавания или шестеро мудрецов (rig pa'i skyes bu thub pa drug). Некоторые
существа покоряются просветленным телом - двенадцатью деяниями и т.д.,
осуществляемыми телом великой заслуги. Другие покоряются просветленной речью -
различными Колесницами, которые являются запредельными, истинно существующими
звуками и словами. Многие покоряются просветленным умом - шестью формами
ясновидения (mngon shes), такими как совершенные активности Самантабхадры. И, кроме
того, многие покоряются невыразимой просветленной активностью, осуществляемой
разными способами, прямо или исподволь. Как сказано в "Аватамсака-сутре (phalро che)":

\begin{verse}
О, благородное дитя, воплощения татхагаты ничем не ограничены.\\
Они заботятся о благе существ, используя те формы,\\
Цвета и названия, которые лучше всего подходят для их усмирения.
\end{verse}

Из них, те, что проявляют двенадцать просветленных деяний, известны как "высшая
Нирманакая". Другие формы, покоряющие других своим состраданием, известны как
"многообразные нирманакаи". Те, что рождаются таким образом, являются эманациями в
физических телах, непосредственно помогающими существам. Таким эманации действуют
на благо других тем, что воплощаются, например, как гигантская рыба во время голода, как
[чудесное?] существо во время эпидемии, или как конь Аджанеябалаха, ускакавший в страну
демонов.\\
\\
Сотворенная Нирманакая проявляется как физические объекты, изображения,
инкрустации, лотосы, исполняющие желания деревья, парки, сады, чудесные дворцы,
драгоценности, корабли, мосты и светильники. Вкратце, все вещи, которые даруются на
благо живых существ, являются благословенными эманациями. Все это возникает из
Пространства и снова растворяется в нем. Способ, которым это происходит, может
выдержать самый детальный анализ. Всеведущий далее объясняет:

\begin{verse}
Когда нет усмиряемого, усмиряющее втягивается в Пространство;\\
Собственное проявление Самбхагакаи втягивается обратно в Дхармакаю.\\
Так же как отражение луны уходит в небо, когда нет отражающей воды,\\
Так же как луна снова исчезает в пространстве, когда приходит время,\\
И так же как молодой месяц не растет и не убывает,\\
Когда есть усмиряемые существа, они постепенно проявляются, как прежде.\\
Таково спонтанное присутствие Плода.
\end{verse}

Как здесь отмечается, когда нет воды для отражения луны, оно естественным образом
втягивается обратно в пространство. И точно также, когда нет покоряемых существ,
подобное луне отражение Нирманакаи, Будда, который проявляется в их восприятии, просто
растворяется в состоянии безмятежности, внутри собственного проявления Самбхогакаи.
Однако, несмотря на использование этого примера, это не означает, что одна вещь возникает
из чего-то другого, а затем растворяется в ней.\\
\\
И точно также Самбхогакая снова растворяется во внутреннем свете Пространства
Дхармакаи. Этот процесс рассматривается как "растворение (thim ра) мудрости обратно в
Пространство", и это похоже на то, как молодой месяц остается в состоянии внутреннего
света. Последователи мадхьямаки объясняют это как медитативное равновесие (тпуат par
gzhag ра) внутри состояния высшего прекращения ('gog ра dam ра), которое возникает в
результате безмятежного пребывания в Пространстве явлений. Они также считают, что
полезная работа осуществляется в восприятии других на основании предыдущих
благопожеланий (smon lam). В "Высшей непрерывности", с другой стороны, утверждается,
что хотя мудрость медитативного равновесия не колеблется, невообразимое количество
полезного приносится живым существам благодаря постмедитативному состоянию. И здесь
говорится:

\begin{verse}
Мудростью считается неконцептуальность,\\
А также следующее за этим достижение.\\
\end{verse}

\begin{siderules}
Поскольку содержание первого раздела объясняется подробно в основном трактате, его
легко понять. Далее, поскольку данное детальное объяснение было бы слишком
пространным, я его не представил. И я также виже, что здесь нет никаких трудных моментов,
которые бы нуждались в разъяснении.
\end{siderules}

\subsection{Уникальное представление мантры}

Что касается чудесного состояния двойной чистоты, в котором отбрасывание и
постижение достигают своего предела, говорится, что в Сутре и Мантре нет разницы между
Состоянием Будды. Такова позиция моего учителя, всеведущего владыки речи, Лонгченпы.
Однако, когда оцениваются качества Пространства относительно глубокого постижения и
активностей иллюзорной Мудрости реализованных существ, неописуемые моменты могут
быть поняты точно, поскольку такова сама природа их знания, делать это со всем, с чем
сталкиваются.\\
\\
Одним таким человеком был беспримерный в объяснении таких тем Юнгтон Дордже
Пал. В традиции мастеров Зур, среди тех, кто практиковал Великого Славного, он составил
трактат, который отделяет идеи Сутры о Состоянии Будды от идей Мантры. Хорошо
известно, что когда доходили до этого текста, то даже несравненные мастера объяснения
предпочитали выказывать скромность. Тем не менее, после тщательного рассмотрения,
представляется возможным предложить анализ этой темы, на основе трактата этого мастера.\\
\\
Что касается Дхармакаи, то он учил, что можно сделать три разделения: разделение
относительно сущности (природы?) (ngo bo), разделение относительно характеристик
(mtshan nyid) и разделение относительно благословений (byin rlabs). Вначале мы имеем
разделение относительно сущности. Дхармакая Причинной колесницы характеристик (rgyu
mtshan nyid kyi chos sku) есть пустотность, отсутствие концептуальных проекций (spros ра).
Дхармакая Колесницы мантр, с другой стороны, предполагает союз проявления и
пустотности (snang stong zung 'jug), как утверждается в "Обширной иллюзии (rgyas ра)":

\begin{verse}
Одушевленный и неоодушевленный мир\\
Проявляются, хотя и не имеют сущности.
\end{verse}

Что касается второго разделения, то объясняется, что Дхармакая Причинной
колесницы впадает в крайность пустотности, тогда как Дхармакая Мантры является
нераздельностью проявления и пустотности и, как таковая, она не впадает в крайность. В
связи с третьим разделением он объясняет, что в результате благословения Дхармакаи в
Причинной колеснице появляются только два Тела-формы (gzugs sku), тогда как в традиции
Мантры благословения нераздельности проявления и пустотности таковы, что могут
возникать пять Тел и любые другие формы.\\
\\
Однако при детальном рассмотрении я не вижу никакой разумной причины проводить
различие относительно сущности Дхармакаи. Причина этого состоит в том, что когда речь
идет об очень тонкой мудрости, которая направлена вовнутрь, но не притуплена, проявление
и пустотность не идентифицируются (ngos bzung med ра) как таковые. Далее, цитата из
"Обширной иллюзии", которая цитируется в данном контексте, на самом деле совершенно
соответствует разделению, сделанному относительно характеристик Дхармакаи.\\
\\
Во втором разделе разбирается разница между двумя Телами форм в этих традициях,
и всего их три. Что касается первой, то в Причинной колеснице характеристик две Тело-
формы возникают вследствие причин и условий, тогда как в Колеснице мантр это не так. И
снова в "Обширной иллюзии" сказано:

\begin{verse}
Поскольку они не зависят от причин и условий...
\end{verse}

Следующий раздел имеет два подраздела, связанные с разницей относительно
Самбхогакаи и Нирманакаи. Что касается Самбхогакаи, то есть две разницы, и первая
связана с разницей в том, чем наслаждаются. Колесница характеристик утверждает, что
наслаждаются позитивными факторами, тогда как негативными - нет. А также есть различие,
делаемое на основе методов, благодаря которым наслаждаются этими факторами. Колесница
характеристик не обладает методами использования негативных факторов, тогда как
Колесница мантры обладает методами использования как позитивных, так и негативных
факторов.\\
\\
Кроме того есть две разницы относительно Нирманакаи. Первая связана с объектом -
существами, которые нуждаются в руководстве. Нирманакая Колесницы характеристик
может лишь усмирять учеников с добродетельным характером, но не с недобродетельным.
Нирманакая Колесницы мантры, с другой стороны, не делает таких разделений. Вторая
разница имеет отношение к методам, которые используются для усмирения учеников. И
снова, Нирманакая Колесницы характеристик не имеет методов для усмирения учеников с
негативными склонностями, тогда как Нирманакая Колесницы мантры обладает методами
усмирения как добрых, так и злых учеников. Поэтому, подытоживается в тексте, Колесница
мантры также может быть понята как высочайшая относительно нераздельности Ваджракаи
и Абхисамбодхикаи.\\

\begin{siderules}
В этом разделе описываются особые моменты метода Мантры относительно темы Состояния
Будды. Есть три традиции разделения Состояния Будды в Колесницах Сутры и Мантры. В
первой утверждается, что кроме различной длинны, два пути Сутры и Мантры в сущности
одинаковы, поскольку оба ведут к Состоянию Будды, как к своему результату. Во второй
утверждается, что без практики пути Тайной мантры Ваджраяны невозможно
актуализировать состояния совершенного и полного Просветления. В третьей утверждается,
что хотя на пути Сутры можно развиваться вплоть до одиннадцатого уровня этой традиции,
то есть до уровня Всеозаряющего света (kun tu 'od kyi sa), который здесь рассматривается как
Состояние Будды, это не может соответствовать подлинному состоянию Единого
Ваджрадхары (zung jug rdo rje 'chang), воплощению океана Тел и Мудростей, о которых
говорится в традиции Мантры. Иными словами, эта традиция разделяет Состояние Будды в
Сутре и Мантре.\\
\\
Если тщательно исследовать, то последние две позиции сводятся к одному и тому же. Что
касается первой позиции, то необходимо принять то, что Состояние Будды - одиннадцатый
уровень Всеозаряющего света, путь за пределами медитации - является абсолютным
состоянием, достигаемым на пути Сутр. И напротив, если путь не ведет к описываемому
результату, то этот путь и его результат идут вразрез, подобно путям к освобождению в
неБуддийских традициях. И это становится неприемлемым искажением Колесницы
характеристик.\\
\\
Что касается второй позиции, то следует принять, что когда достигнуто Состояние Будды,
связанное с путем Сутры, тем не менее, необходимо вступить на путь Мантры, чтобы
достичь состояния Единого Ваджрадхары. С другой стороны, можно пологать, что
Состояние Будды на пути сутр является вершиной пути и не использовать путь Мантры.
Проблема, однако, состоит в том, что ограничиваются двумя абсолютными Колесницами и
путь Мантры не может быть даже включен в три Колесницы.\\
\\
Обе позиции поддерживаются многими мудрыми и реализованными индивидуумами. В
частности, текстовую опору для разделения Состояния Будды на Сутру и Мантру можно
найти в "Магической сети (sgyu 'phrul)" и в "Произнесении имен Манджушри (mtshan brjod)".
А также объяснению этого большое внимание уделяет индийский учитель
Буддхаджнянапада. В свете того, что также большое количество других текстовых опор, это
положение не следует рассматривать как беспочвенную и лишь условную позицию.
\end{siderules}

\subsection*{Завершающие строфы и колофон}

\begin{verse}
Подобно прочной ступени к Пространству \\
высшего и неизменного великого блаженства,\\
Украшенной золотыми драгоценностями \\
детальных наставлений о смысле тантр,\\
Это - собрание хороших советов, \\
ускоряющих путь к Акаништхе,\\
Несет ответственность за слова и смысл высочайшей из Колесниц.
Воплощение всех Будд, Владыка Уддияны, Махапандита Вималамитра,
Всеведущий Владыка Дхармы и сама жизненная сила \\
Школы старых переводов, бодхисаттвы Зурчен и Зурчунг;\\
Когда большинство традиций учения этих мастеров \\
были едва живы в эти темные времена,\\
Они были освещены светом, исходящим \\
из солнца Ранджунга Дордже Кхенце.\\
Эта сокровищница исполняющих желания драгоценностей \\
очищает сущностные и ключевые наставления,\\
Собрания тантр и садханы, полностью и безошибочно.\\
Те, кто следуют по этим стопам и направляются к пяти Телам,\\
Определенно, являются неоспоримо счастливыми гостями.\\
Безмятежная река этого изначального собрания заслуги,\\
Сопровождаемая последовательностью волн трех превосходств,\\
Осуществляет передачу знания Манджукумары.\\
Благодаря этому, пусть все достигнут уровня Самантабхадры!
\end{verse}

\small
Чагсам Ригдзин - высочайшая реинкарнация Еше Лхундруба Палзангпо и тот, кто
достиг благого ока разума, широко видящего то, чему татхагаты учат так хорошо. И все же
он пребывает без тщеславия относительно учений и тех, кто поддерживает их с великой
заботой и уважением. С большим упорством он просил, чтобы я составил этот текст
"Лестница в Акаништху: Наставление по Стадии Зарождения и Йоге божества", сопровождая
свою просьбу подношением одежды вышитой золотом и пары разноцветных шарфов.
В ответ на эту просьбу бутон лотоса моего ума мгновенно раскрылся в свете чудесных
лучей любви и мудрости, которые были испущены славным Падмасамбхавой и его супругой.
В результате понимание всех явлений возникло естественно, и я смог бесстрашно оставаться
перед лицом подлинной истины. Я, Рангджунг Дордже Джигме Лингпа, практикующий
Великого совершенства, также известный как Лонгчен Намке Налджор, затем составил этот
текст в Церинг Джонг. Он был записан посреди леса просветления в ретритном центре,
известном как Падма Осел Тегчог Линг, в чудесном месте для духовной практики, в храме,
охраняемом татхагатами пяти семей. Во время написания этого объяснение мне было
тридцать девять лет. Работа была завершена в четверг, в лунный месяц танцора (gar mkhan
gyi khyim zla), в год свиньи (1767), во время восхождения дома льва (seng ge'i ta tal la 'char
ba).

\begin{center}
БЛАГО БЛАГО БЛАГО
\end{center}

\begin{siderules}
Если добрая удача слушания лишь единого слова высших учений,\\
Содержащихся в линии Всеведущих, выходит за пределы мыслимого,\\
Тогда эти определенные идеи, основывающиеся на объяснении, полемике и составлении,\\
Появляются в результате силы накопления заслуги в течение многих кальп.\\
\\
И хотя сам я не обрел не малейшей уверенности\\
Посредством изучения, созерцания и медитации,\\
Я написал это, чтобы ознакомиться с этими учениями,\\
И чтобы помочь немногим удачливым и благонравным.\\
\\
Знающие ученые, которые многое изучили,\\
Йоги, опытные в медитации,\\
И те счастливцы, обладающие сущностными наставлениями,\\
Пожалуйста, исследуйте этот подход и устраните нечистое своей любовью.\\
\\
Так, я призываю всех ученых, чье описание сотен текстов\\
Может прояснить объяснения Всеведущего нашей традиции,\\
Отсечь ошибки, которые затемняют завесами\\
Неведения, непонимания и частичного понимания.\\
\\
Если здесь есть ясное объяснение, то я посвящаю\\
Накопление этой заслуги процветанию Трех Драгоценностей,\\
А также тем, кто проясняет сущностные учения Могучего,\\
А также постоянно поддерживает, объясненяет и практикует их.\\
\\
В ответ на просьбу Лхундруба Дордже, обладающего ясным разумением, врожденным и\\
приобретенным в результате обучения, я, беспечный бродяга Any, записал то, что пришло на\\
ум. Пусть это принесет великое благо!\\
\end{siderules}

\begin{center}
САРВА МАНГАЛАМ
\end{center}

