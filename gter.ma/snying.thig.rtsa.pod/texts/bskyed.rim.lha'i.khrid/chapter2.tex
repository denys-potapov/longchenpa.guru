
\section{Безошибочное условие}

\subsection{Упражнение на Пути, \\ соответственно своим способностям}

1) Сосредоточение ума на божестве (lha la sems bzung ba);\\
2) Исправление ошибок изменений ('gyur ba'i skyon bcos ра);\\
3) Отделение от божества (lha dpral ba);\\
4) Привнесение божества на путь (lha lam du slang tshul);\\
5) Смешивание ума с божеством (lha rang sems dang bsre ba);\\
6) Объединение божества с реальностью (lha don las byar ba);\\
7) Привнесение переживаний на путь (spyod yul lam du khyer ba).\\

\subsubsection{Сосредоточение ума на божестве}

Те, кто обладают высшими умственными способностями (blo chen), должны очищать
свои способности посредством практики Стадии Зарождения без цепляния и фиксации. При
таком подходе проявление божества и его украшений визуализируются как полностью
совершенные (yongs sur dzogs ра), ясные и отчетливые с самого начала. Такова форма
великой Мудрости, союза Зарождения и Завершения. Выходя за пределы определенной
сущности с конкретной природой, она [форма-проявление Мудрости] проявляется ясно, хотя
и пуста от сущности (ngo bo). Иными словами, ясность и пустота неразделимы. Подобно
отражению луны в озере, её природа отчетливо проявляется для глаз учеников, хотя на
самом деле является пустой.\\
\\
Те, у кого средние умственные способности (blo 'bring), должны начинать свою
медитацию с мгновенного вспоминания всего проявления тела божества. Следующая стадия
— это медитация на ясном проявлении головы. Когда её видение становится устойчивым,
переходят к медитации на правой руке, левой руке, на туловище, на левой и правой ноге, и,
наконец, на всей форме божества и на его троне. Тренировка Стадии Зарождения
иллюзорного и ясного проявления препятствует появлению воззрения нигилизма (chad lta).
Когда устают, то необходимо вспоминать чистоту и развивать (очищать) способности (rtsal
sbyangs) по отношению к сущности этого процесса в ваджраподобном самадхи. Этот
ключевой момент (gnad) препятствует появлению воззрения постоянства (rtag lta).
Для начинающих с малыми умственными способностями может быть трудно
визуализировать любым из этих способов. Когда не очень хорошо знакомы с этим
процессом, необходимо развивать способности посредством использования постоянной
формы (rtag pa'i sku). Возьмите освященныйи правильно сделанный образ божества йидама,
изображение или статуэтку, выполненные искусным мастером, и поместите её перед собой.
Без специальной медитации рассматривайте эту форму с головы до пят, не мигая. Это
называется "вспомогательной практикой помещения внимательности в движение". Вначале
будут возникать множество обусловленных мыслей ('du byed kyi rtog tshogs), что является
переживанием движения (g.yo ba'i nyams), о котором говорят, как о "подобном воде,
падающей со скалы".\\
\\
Однако, во пределенный момент это меняется, когда развертываются привычные
тенденции, связанные с проявлением божества. Когда это происходит, его проявления
возникают отчетливо, как объекты в уме (yid yul du), даже когда глаза закрыты. Это
называется" переживанием достижения (thob pa'i nyams)", которое сравнивается с вором,
который достает то, что спрятано в сосуде. Иными словами, в этот момент на самом деле не
знают природы божества, как есть (ji bzhin).\\
\\
Подобная визуализация запечатывает фиксацию на обычных проявлениях (tha mal gyi
snang zhen la rgyas btab). Эта визуализация не должна быть плоской, как рисунок, или же
выпуклой и твердой, подобно барельефу. Это должна быть иллюзорная фигура, пустое
проявление, свободное от любой вещественности, без плоти, костей и внутренностей. Можно
развивать способности в этом процессе, рассматривая примеры без упречной хрустальной
вазы и радуги, которые полностью прозрачны, сверкающие и сияющие (zang thal gsal 'tsher
bkrag mdangs). Данная визуализация должна быть подобна отражению в зеркале.\\
\\
В начале возьмите образ, который вы используете как опору для визуализации (dmigs
rten) и смотрите на него долгое время, медитируя лишь недолго. Когда ваша визуализация
станет более ясной, вы постепенно сможете сократить время всматривания в образ и
продлить время медитации.\\

\subsection{Исправление ошибок, \\приводящих к изменениям}

Есть семь общих ошибок, которые могут возникнуть на данном этапе:\\

\begin{tabular}{ll}
1 & отвлечение от объекта сосредоточения (dmigs ра brjed ра);\\
2 & оцепенение или ленность (le lо);\\
3 & страх недостичь цели (mа grub kyi sdogs ра);\\
4 & вялость (bying ba);\\
5 & возбужденность (rgod ра);\\
6 & неудовлетворенность, даже если божество проявилось \\
  & отчетливо (lha gsal bzhin du chog mi shes pa'i rtsol ba);\\
7 & безразличие, даже если визуализация не отчетлива \\
  & (mi gsal kyang btang snyoms su 'jog pa'i mi rtsol ba).\\
\end{tabular}

А также есть двенадцать ошибок, которые приводят к изменениям в визуализации:\\

\begin{tabular}{ll}
1  & нечеткость (rab rib);\\
2  & блеклость (mog mog);\\
3  & затемненность (grib mа lta bua'm mun pa);\\
4  & изменения размеров (sku' tshad 'gyur ba);\\
5  & изменения внешнего вида (cha lugs 'gyur ba);\\
6  & изменения очертаний (формы) (dbyibs 'gyur ba);\\
7  & изменения количества (grangs 'gyur ba);\\
8  & изменения расположения (bzhugs tshul 'gyur pa);\\
9  & проявление лишь как цвета (kha dog 'ba' zhig snang ba);\\
10 & проявление лишь как очертания (dbyibs 'ba' zhigs nang ba);\\
11 & постепенное исчезновение (rim par nub pa);\\
12 & неполное проявление (rnam 'gyur ma rdzogs pa).
\end{tabular}

Противоядия (gnyen ро) от этих ошибок следующие: Противоядием от отвлечения
является устойчивая внимательность (dran ра brten). Если впадают в оцепенение, то
развивают веру и усердие (dad ра dang brtson 'grus). Когда пребывают в сомнении, то
сосредотачиваются на Пространстве (dbyings), а такжео стаются в светлом и прохладном
месте, если чувствуют вялость. Когда возникает возбужденность, зарождают сожаление (skyo
shas) и направляют взгляд вниз. Если неудовлетворены, то расслабляютум (blo lhod), а если
безразличны, то проявляют усердие. Точно также, если визуализация нечеткая,
неопределенная или затемненная, то ставят перед своими глазами искрящийся кристалл ('ja'
shel). Затем смотрят на опору для медитации и проверяют её очертания. После этого её
визуализируют в своем уме.\\
\\
Если, с другой стороны, размер тела, его проявление, положение или очертания
меняются, то есть различные вещи, которые можно сделать. Можно медитировать на том,
что тело божества огромное и твердое, что олени резвятся на его руках, ногах, на пальцах
ноги рук, а также, что голуби влетают и вылетают из его ноздрей. В частности, в связи с
мирными божествами соответствующие проблемы могут быть устранены посредством
визуализации этих божеств, как обладающих природой, имеющей девять черт (tshul dgu).
Относительно этого в "Устрашающей вспышке молнии (rngam glog)" сказано:

\begin{verse}
Каждая и хвысшая форма\\
Обладает девятью чертами.\\
Они мягкие, стройные,\\
Крепкие, гибкие и юные,\\
Ясные, сияющие, привлекательные\\
И пылающие интенсивным присутствием.
\end{verse}

Гневные образы должны представляться как имеющие девять поз. Как сказано в "Херука галпо тантре":

\begin{verse}
Очаровывающая, героическая и ужасающая,\\
Смеющаяся, свирепая и устрашающая,\\
Сострадательная, угрожающая и успокаивающая,\\
Прими эти девять танцевальных поз.
\end{verse}

Следующий отрывок из "Гуру магической сети" обращается к проблеме колебаний среди множества божеств:

\begin{verse}
Даже многочисленные излучения и поглощения ('phro 'du) собрания божеств
Являются великим чудесны мпроявлением своего собственного ума.
\end{verse}

Как здесь утверждается, в этом процессе вы можете очистить свои способности,
представляя, как формы свиты возникают из чудесног опроявления одного главного
божества. С другой стороны, если появляется только цвет, то нужно представлять очертания.
Если же появляются лишь очертания, то следует превратить их в различные цвета. Когда
визуализация современем исчезает, медитируй, что лицо и руки очень твердые (rags ра), и
если она является неполной, то сосредотачивайся на визуализации того, чего не достает.

\begin{siderules}
Эта тема делится на семь категорий, первую из которых легко понять. Что же касается
второй, то мы находим двенадцать недостатков, которые приводят к изменениям
визуализации. Первое — это нечеткость, связанная с тем, что проявление божества едва
видно, его очертания, лицо, руки и другие части неотчетливы. Следующая — блеклость, это
недостаточная красочность визуализации. Когда это происходит, то белый, красный, синий и
другие цвета визуализации недостаточно интенсивные и насыщенные, подобно ясному и
сверкающему свету тысячи сияющих солнц. Вместо этого они бледно-белые и бледно¬красные.
Также может быть, что фиксируя свой ум на визуализируемом проявлении
божества, затем ненность препятствует видению ума. И даже если это не происходит,
проявление расплывчато и неясно, подобно туманному и нечеткому восприятию, когда
толщина рук и ног, а также ширина лица и глаз проявляются неотчетливо.\\
\\
Другая проблема связана с изменение мразмеров божества. Вы можете визуализировать
не что размером с маленькое семечко, например, но когда проявляется, то это уже имеет
размер большого пальца. Это также может быть связано с формой мирного или гневного
божества, лишенного пропорций, как на плохом рисунке. Когда случаются изменения
внешнего вида, то девять поз, украшения мирных божеств и другие подобные
характеристики не соответствуют должной визуализации. Другая проблема возникает, когда
божество изменяет форму, например, вы визуализируете мирное божество, чье лицо имеет
форму горчичного семени, а в итоге его очертания приобретает форму привлекательного
женского божества. Изменение количества происходит, когда три лица становятся четырьмя,
или когда появляется множество главных божеств, а не лишь несколько. Изменение
расположения предполагает изменение позиций лотоса, полу-лотоса и т.д.\\
\\
Возможно проявление лишь цвета. Иными словами, есть различия белого, красного и других
цветов, однако толщина и другие характеристики, связанные с очертаниями визуализации не
четки. Кроме того может быть проявление только очертания, а не цвета. Когда происходит
постепенное исчезновение, то визуализация становится все менее ясной, хотя вы и не на
Стадии Завершения. И не полное проявление происходит, когда могут визуализировать
каждую часть тела — голову, руки и т.д., главного и второстепенных божеств, но не их в
целом.\\
\\
Девять черт мирных божеств (zhi ba'i tshul dgu) являются противоядиями (gnyen) от этих
проблем. Формы божеств, лицо, руки и т.д. должны быть мягкими (mnyen pa), а не жесткими,
как кости или деревья. Их тела должны быть пропорциональными, постепенно сужающимися
к талии. Их плоть нед олжна быть дряблой, но упругой. Они должны быть гибкими, в том
смысле, что их суставы и члены подвижные. Их тела должны выглядеть здоровыми, а их
кожа нежная и юная. Их цвета должны быть четкими и ясными, и они должны излучать свет
в том смысле, что они должны излучать безграничный свет. Их формы должны быть
величественные и украшенные знаками высших существ, и, следовательно,
привлекательными. И, наконец, у нихдолжно быть интенсивное присутствие, затмевающее
все своим блеском.\\
\\
Гневные божества также обладают девятью качествами. Их выражение (nyams) страсти
должно быть пленяющим (sgegs ра), их выражение гнева должно быть героическим (dpa' ba),
их выражение неведения должно быть устрашающим (mi sdug ра). Таковы три выражения
(экспрессии) тела. Они должны громко смеяться (gad mо) "ха-ха" и "хи-хи". Они должны
свирепо кричать — "схвати, бей!". Они должны грозно реветь, подобно раскатам тысячи
громов и разрывам тысяч молний. Таковы три их выражения речи. Также они должны
сострадательно брать заблудших существ и миры под свою опеку. Они должны запугивать,
покоряя варваров своим гневом. И они должны быть мирными в том смысле, что они
воспринимают все, как имеющее единый вкус Дхарматы. Таковы девять танцевальных
выражений-экспрессий.
\end{siderules}

\subsubsection{Отделение от божества}

Третья тема связана с тем, что называется "отделением от божества". В интервалах
между этими медитациями нужно включать периоды медитации на божестве с
неконцептуальным созерцанием (mirtog ра nyam par 'jog ра). Это ослабляет ощущение
усталости от практики, которое может появиться. Затем снова медитируют только на
божестве, визуализируя ясно и безошибочно, без концепций.\\

\subsubsection{Привнесение божества на путь}

Четвертая тема связана с тем, как привнести божество на Путь, или, иными словами,
как очистить (если тиб. sbyang ba, то это-тренировать) свои способности, когда
становятся последователем этих практик. Визуализируй божество стоящим, сидящими
лежащим. Представляй его стоящимна макушке головы, лежащим на спине, лицом вниз,
находящимся далеко, близко, внутри горы, в глубине океана, массивным, как гора Сумеру,
или же маленьким, как пылинка. Если ты можешь медитировать так, тогда, достигнув
медитативного переживания, подобного реке, ты привнесешь божество на путь.\\

\subsubsection{Смешивание своего ума с божеством}

Пятая тема называется "смешиванием своего ума с божеством". Чтобы устранить
любые двойственные фиксации, которые могут быть в отношении медитирующего и
медитации, упражняйся в нераздельном смешивании своего ума с формой божества. Когда
ты сможешь оставаться в полной и совершенной форме божества так долго, как захочешь, а
также не отвлекаясь на мысли, тогда ты достигнешь устойчивого переживания, подобного
горе.\\
\\
И далее наступит момент, когда ты сможешь визуализировать божество очень
детально, вплоть до пор на его теле и зрачков глаз. Также, тогда ты не будешь подвластен
никаким внешним влияниям и сможешь медитировать день и ночь на мандале своего
божества. На этому ровне совершенного мастерства твоя способность станет совершенной.
Есть восемь мер ясности и устойчивости, которые указывают на достижение или не
достижение этого. Четыре меры ясности — это сияние (gsal le), чистота (sang nge), блеск (lhag
ge) и отчетливость (lhang nge). Четыре меры устойчивости (brtan ра) — это неподвижность (mi
g.yo), неизменность (mi 'gyur), совершенная неизменность (mngon par mi 'gyur) и всемерная
изменчивости (cir yang bsgyur). Когда эти восемь мер ясности и устойчивости достигнуты,
выходят на уровень, известный как "переживание совершенства (mthar phyin)". Тогда
смешивают проявления с мандалой божества.\\

\begin{siderules}
Третью и четвертую темы легко понять. В пятой мы видим восемьмерясностии
устойчивости. Первая — это сияние, которое относится к ясному и отчетливому проявлению
того божества, на котором медитируют, вплоть до зрачков его глаз. Следующая мера — это
чистота, состояние, в котором осознавание имеет смысл жизненной силы (seng bag). Это
ясное, пустое и отчетливое, а не притупленное и лишенное ясности восприятие. Если
божество, на котором вы медитируете, это лишь мертвое и твердое проявление, а не похожее
на радугу, как должно быть, тогда каждая его деталь должна быть охвачена мудростью
всеведения (thams cad mkhyen pa'i ye shes), вплоть до пор и волосков на теле. Когда это есть,
то возникают и сверкающе присутствуют сотни качеств, связанные с ясностью органов
чувств, чтои является блеском. Форма божества также должна быть интенсивно отчетливой.
Иными словами, вы не должны мыслить об этом, выводя (rjes dpag) присутствие и
проявление. Вместо этого, они должны возникать в уме непосредственно (mngon sum), с
отчетливой ясностью. Таковы четыре меры ясности.\\
\\
Далее следуют четыре меры стабильности. Первая — это неподвижность, которая означает,
что медитация не может быть поколеблена таким общими недостатками, как забывчивость и
лень. Вторая — это неизменность вопреки недостаткам изменений визуализации, таким как
расплывчатость и нечеткость. Когда визуализация больше не появляется лишь на короткие
периоды времени, но может поддерживаться день и ночь, без отвлечения даже на самые
тонкие мысли, она становится совершенно неизменной. И, наконец, когда вы медитируете на
таких факторах, как цвет божества, его лицо и руки, движения, испускание и втягивание
лучей света, а также все возникает как полагается, то практика становится всемерно
изменчивой.\\
\\
Вобщем, способность сохранять ясность и спокойствие, когда практикуют медитацию на
божестве, — это позитивное качество. Тем не менее, независимо от того, что вовремя
медитации форма божества является неясной ил иже отчетливо ясной, хоть и не медитируют,
есть ошибка цепляния за визуализацию. Поэтому, когда осваивают ясное проявление
визуализации, необходимо упражняться в стадии растворения (bsdu rim).\\
\\
Сегодня, много людей не практикуют стадию растворения, кроме как в коротком завершении
медитации. Они говорят разное: "Вы никогда не должны терять из виду три мандалы,
поэтому не следует делать стадию растворения ит.д." Это указывает на то, что они не только
непонимают ключевые моменты Стадий Зарождения и Завершения, но даже не видят
ошибочность концептуальной фиксации по отношению к высшему божеству. Очевидно, что
это признак неведения. Именно поэтому учат стадии растворения.
Стадия растворения очищает привычные тенденции, связанные со смертью и устанавливают
причинную связь с Дхармакаей. На пути Стадии Завершения иллюзорное тело растворяется
в свете. Стадия растворения не только готовит практикующего к этой стадии практики, но
также к проникновению ключевых моментов (gnad dub snun) ваджрного тела, а также к
стягиванию каналов, ветров и капель-тигле в центральный канал. Вобщем, только это и
порождает все просветленные качества пути Ваджраяны.\\
\\
Был ли развит символический или истинный Ясный свет в своем собственном состоянии
бытия, есть десять знаков, которые появляются, когда элементы последовательно
растворяются, в то время как из природы стадии растворения (thim rim) возникает Ясный
свет трех проявлений. Когда он появляется, все обычные ложные проявления втягиваются в
пространство (klong), и три проявления растворяются в Ясном свете. При этом есть лишь
возникновение Дхарматы, Мудрости, свободной от измышлений.\\
\\
Когда это происходит, некоторые практикующие должны больше внимания уделять Стадии
Растворения, а не Стадии Зарождения. Это актуально для тех, кто хочето братиться к
добродетельным устремлениям вов ремя состояния глубокого сна и сновидений, а также для
тех, кто днем вовлечен в практику введения ветра-ума в центральный канал. Те, кто хотят
практиковать особую практику Стадии Завершения, которая актуализирует [делает явными]
иллюзорное телои Ясный свет, также должны сделать упор настадии растворения. И это же
должны сделать те, кто хотят появиться в божественном теле единства тренировки (slob pa'i
zung 'jug gi lha sku), когда отбрасывают свое обычное тело. Это также относится к тем, кто
стремится к совершенству этого процесса, к актуализации состояния единства без
тренировки. Это состояние достигается благодаря растворению в Ясном свете ментального
тела, которое опирается на привычные тенденции неведения. И, наконец, есть те, кто
заканчивает свои дни, так и не достигнув Просветления, а также те, хочет появиться в
объединенной Самбхогакае, пребывая в Ясном свете первого промежуточного состояния, в
состоянии (dgongs ра) Дхармакаи. Эти практикующие также должны делать больший упор
на фазе растворения, чем на Стадии Зарождения.\\
\\
Кроме того могут быть и те, кто лишь хотят осуществить низшие [тиб. smad las] активности,
подобные покорению духов и дарованию защиты. Для этого они сосредотачиваются
исключительно на визуализируемых проявлениях божества йидама, а также на
распространениии собирании лучей света. Такие практикующие могут почувствовать
удовлетворенность от своих визуализаций, а также породить гордость. Однако такой подход
не отличается от не-Буддийского. В итоге — они будут переживать последствия своих
негативных действий.\\
\\
Некоторые могут возразить, что "это может быть и так, но для чего смешивать нашу
терминологию с терминологией новых школ, используя термины "иллюзорное тело" и
"Ясный свет". — Какой стыд говорить такие вещи! Это последователи Нингма, которые даже
невидели определенные классы тантр, такие как "Тантра совершенной тайны (gsang ba yongs
rdzogs)" и "Херука Галпо-тантра". Такие люди ничего не знают об основных текстах,
связанных с практиками трех внутренних тантр!\\
\\
Тем не менее, считается, что характерная черта (khyad chos) школы Нингма Ранних
Переводов (snga' 'gyur), это утверждение, что цитадель постижения захватывается в
пространстве ваджрной вершины, на пике всех колесниц (theg rtse rdo rjer tsemo'i dbyings su
dgongs pa). В подобном утверждении подразумевается то, что необходимо отбросить все
формы ментальной деятельности, даже Стадию Зарождения и другиедесять природ. Иными
словами, необходимо пребывать в обнаженной открытости пустого осознавания (rig stong
zang thal), как объясняется Дриме Озером в "Сокровищнице драгоценного Дхармадхату (chos
dbyings rinроche'i mdzod)". Следуй же по стопам этого совершенного Будды и ябуду рад!
\end{siderules}

\subsubsection{Объединение божества с реальностью}

Вы не должны успокаиваться на простом смешивании своего ума с божеством. Чтобы
связать божество с его истинной природой (don), необходимо понять, что это именно твой
собственный ум, вместе с собранием восьми сознаний, возникает как форма и мудрость
божества. В этом присущем состоянии самоосознавание (rig ра) является пробужденным
умом (бодхичиттой), результирующей формой божества. Будучи разнородными, они
созревают как один вкус, через нераздельность Стадий Зарождения и Завершения - великое
Тело Мудрости.\\

\subsubsection{Четыре гвоздя, \\скрепляющих жизненную силу}
Но, если этот момент непонят, того Стадия Зарождения не будет отличаться от
формы накопления привычных тенденций, которые закрепят обычное состояние
существования. Тогда ты отвернешься от пути Освобождения и твое достижение сведется не
более чем к перерождению злым духом или демоном, подобным Рудре! Вот почему учат, что
"четыре гвоздя, скрепляющих жизненную силу (srog sdom gzer bzhi)" имеют огромную
важность. В тридцать девятой главе "Тантры совершенной тайны (gsang rdzogs)" сказано:

\begin{verse}
Мудрость или мирское, если четыре гвоздя,\\
Скрепляющих жизненную силу, не вбиты,\\
Практика никогда не будет Плодотворной.\\
Способо владеть жизненной силой всех Славных,\\
Состоит в понимании одного момента и обретении жизненной силы всего.\\
\end{verse}

\begin{verse}
Также как Рахула проглатывает солнце на небе,\\
Не гоняясь за отражениями в тысячах водоемов,\\
Для обретения жизненной силы их всех\\
Четыре гвоздя, которые скрепляют жизненную силу, очень важны.
\end{verse}

"Гвоздь самадхи (tin nge 'dzin)" позволяет превратить сильную привязанность по
отношению к своему обычному телу в форму божества. Это происходит благодаря
однонаправленному сосредоточению (sems rtseg cig) на форме божества, как на опоре,
используемой для визуализации, а также благодаря дальнейшему его развитию. В тантре
сказано:

\begin{verse}
Гвоздь самадхи, это однонаправленное сосредоточение ума\\
На Теле божества и на капале без отвлечения.\\
Овладей тремяо бъектами и встреться\\
С Великими Славным лицом к лицу.\\
Иначе херуку не увидишь.\\
Без гвоздя постоянного самадхи\\
Никогда не узреть Просветленного Тела.
\end{verse}

Тремя объектами овладевают следующим образом. В начале божество ясно
присутствует как объект мысли (bsam pa'i уul), затем ясно присутствует как видимый объект
(mthong ba'i yul) и, наконец, оно ясно присутствует как осязаемый объект (reg pa'i yul).
Божество является умом, созревшим как чудесная форма (sku), встреча с которой называется
"совершенной визуализацией". В "Советах Велико славного (dpal chen zhal lung)" сказано:

\begin{verse}
Все формы божеств мудрости\\
В начале проясняют как объекты ума.\\
Затем развивают силу их актуального проявления,\\
А затем, благодаря силе совершенно чистого ума,\\
Их природа проявляется как чувственный объект.\\
\end{verse}

\begin{verse}
И, наконец, две истины становятся нераздельными;\\
Достигается совершенная гибкость тела и ума,\\
А божество ясно проявляется, как осязаемый объект.\\
Так ты преодолеваешь явное проявление нечистого тела.
\end{verse}

Что касается "гвоздя сущностной мантры (snying ро sngags kyi gzer)", то следует
однонаправленно сосредотачиваться, когда коренная мантра либо вращается вокруг
жизненной силы в Сердце (thugs srog), либо вращается туда и обратно. Затем, когда
подсчитывается мантра, выдох, вдох и задержка дыхания очищаются как сущность
просветленной речи. В той же тантре сказано:

\begin{verse}
В качестве гвоздя сущностной мантры повторяй мантру\\
Главного божества всего собрания,\\
Как корневую мантру окружающую жизненную силу в Сердце ХУМ.\\
Визуализируй излучениеи поглощение, а также повторяй мантру.\\
\end{verse}

\begin{verse}
В это время появляются слава, процветание и пророчества.\\
Потенциальная сила станет совершенной, а божества соберутся.\\
Без совершенства Приближения посредством гвоздя сущности жизненной силы\\
Божества и связанные обетом Дамчены не появятся.
\end{verse}

Третье — "гвоздь неизменного понимания (Состояния?) (dgongs ра mi 'gyur ba'i gzer)".
Неважно как вы практикуете, с Великим-Славным [Херукой] или же слюбыми другими
мирными и гневными божествами, вы должны постичь, что божество есть не что иное, как
ваш собственный ум. Вместе с этим постижением божество созревает в свою сущность,
становясь одним вкусом с Дхарматой (Реальностью) (chos nyid). И эти два становятся великим
равенством. В тантре сказано:

\begin{verse}
В качестве гвоздя неизменного постижения Великославный [Херука],\\
А также все собрание Мирных и Гневных божеств\\
Являются ничем иным как собственным умом.\\
Даже Рудра является ничем иным как умом.\\
\end{verse}

\newpage

\begin{verse}
Будучи пустым, сам ум есть Дхармакая.\\
Будучи ясным и отчетливым, он — Самбхогакая.\\
И, проявляющийся естественным образом, многими способами, он — Нирманакая.\\
Проявления — это мужские формы, а пустотность — женские.\\
\end{verse}

\begin{verse}
Нераздельность проявленияи пустотности — это Тело Мудрости.\\
Невообразимые мысли, а также воспоминания\\
Являются полностью совершенными, свитой Славного.\\
Они не описуемы и просветлены изначально,\\
\end{verse}

\begin{verse}
Поскольку они никогда не отклоняются от этой природы.\\
Без гвоздя неизменного постижения,\\
Достижение ('grub) Великого Славного\\
Не приведет к высшему Достижению,\\
Лишь к обычному, подобному Рудре.
\end{verse}

Четвертое — "гвоздь излучения (испускания) и поглощения ('phro 'du'i gzer)". Эта
великая чудесная иллюзия ('phrul chen ро) предполагает освоение каждой из четырех видов
просветленной активности (phrin las). Это также необходимо, как белая бура для алхимии
(bzhu brtul). В тантре дается следующее объяснение:

\begin{verse}
В качестве гвоздя излучения и поглощения\\
Визуализация используется для трансформации.\\
Зная это, все желания осуществляются.\\
Благодаря самадхи излучения и поглощения.
\end{verse}

\begin{verse}
Потенциал ума приводится к совершенству,\\
И все, что представляется, — появляется.\\
Силой благословения мантр и мудр,\\
Все активности, помогающие и вредящие, будут осуществлены,
\end{verse}

\begin{verse}
Умиротворяющие, увеличивающие, \\
контролирующие и гневные,\\
Подобно тому, как восхваления \\
и подношения одной драгоценности\\
Приводят к обстоятельствам \\
осуществления всех желаний.\\
Если же гвоздя различных \\
излучений и поглощений нет,
\end{verse}

\begin{verse}
Тогда [ни?]་один объект сосредоточения\\
не приведет к цели.\\
Но если ты знаешь, как переходить \\
от распространения к собиранию,\\
Тогда даже низший демон тедранг\\
Будет выполнять беспрепятственную активность \\
и осуществлять все действия,\\
Поскольку все является чудесным творением ума!
\end{verse}

Когда основные моменты этих наставлений о четырех гвоздях усвоены,
Просветленное Тело будет непосредственно воспринято, и ты стяжаешь его жизненную
силу. Обрети истинную речь, и ты стяжаешь жизненную силу просветленной Речи. И когда
будет контролироваться сама Реальность, ты стяжаешь жизненную силу просветленного
Ума. Твои трое врат соединятся с просветленным Телом, Речью и Умом мирных и гневных
сугат, и многообразие станет одним вкусом (rо gcig). Сделай это, и ты обретешь
тысячекратную жизненную силу Сансары и Нирваны. Так ты очистишь врожденный
потенциал (rtsal) (энергию?) Мудрости и освоишь различные виды просветленной
Активности.\\

\begin{siderules}
Пятая тема объяснена ясно. Шестая тема касается связи божества с реальностью, которая
также называется "развитие (bogs 'don) Стадии Зарождения" и "связь посредством
активности ближайшей причины (nye rgyu spyod pas mtshams sbyar)". Практика сущностных
моменто в медитации Стадии Зарождения, как было описано выше, предполагает вхождение
в йогические практики с особыми временными периодами и количеством. Это позволяет
выйти за пределы мирских путей и получить непосредственную связь с путем видения,
высшим духовным осуществлением Великой печати (phyag rgya chen ро mchog gi dngos grub).
Именно это подразумевается под термином "выход за пределы посредством поведения".
В "Благой колеснице (shing rta bzang ро)" сказано, что, несмотря на высший путь Стадии
Зарождения и другие [практики], сами по себе они не приводят к Освобождению, поскольку
необходимы другие факторы, развивающие практику. Таким образом, хоть и обладают
восемью мерами ясности и устойчивости на пути Стадии Зарождения, но если не используют
сущностные наставления (man ngag) четырех гвоздей, которые скрепляют жизненную силу
для развития ее посредством поведения, тогда будет невозможно достичь результата этого
процесса, состояния видьядхары.\\
\\
Поэтому в контексте связи божества с реальностью вы должны медитировать на пути
определенного совершенства (lam nges rdzogs). Следовательно, нужно осваивать различные
стадии, от самадхи великой пустотности до сложной формы полного круга мандалы, а также
необходимо достичь совершенства восьми мер ясности и устойчивости. Это называется
"путем с определенного поведения с периодами и количеством (spyod pa'i dus grangs nges pa'i lam)".
Как сказано:

\begin{verse}
Достигают могущества и высшего состояния\\
В течение шести, четырнадцати или шестнадцати месяцев.\\
\end{verse}

Таким образом, посредством преобразующих ритуалов (spog cho ga) просто невозможно не
достичь уровня видьядхары в течение шести месяцев.\\
\\
И даже если такое невозможно, сказано, что те, кто посвящают себя практике, поскольку их
путь также нуждается в ясной цели для учений, в которые они вовлечены. Такой тип
практикующих в начале развивает правильное понимание ключевых моментов относительно
Стадии Зарождения и завершения, а затем, с самого начала они связывают свою практику с
подходом и осуществлением, опираясь на сущностные наставления четырех гвоздей,
которые скрепляют жизненную силу. Таким образом, в нашем бытии, соответственно
умственным склонностям индивидуума, рождаются различные качества.\\
\\
Все остальное равносильно отходу от основных моментов Зарождения и Завершения. Просто
называть свою практику "приближением и достижением" и оставаться в ретрите на
протяжение годов не приведет ни к чему, кроме трудностей. Начитывание сотен миллионов
мантр не вызовет даже медитативного тепла (drod) обычных качеств, которые являются
знаками развития на пути. Иными словами, если сущностные моменты Пути не
принимаются во внимание, то усердие не приведет ни к чему, кроме погони за миражем.\\
\\
В данном контексте гвоздя сосредоточения (самадхи) на божестве упоминается "освоение
трех объектов (yul gsum gyad du gyur pa)". И втексте сказано: Вначале божество ясно
присутствует как объект мысли (bsam ра'i yul), затем ясно присутствует как видимый
объект (mthong bа'i yul) и, наконец, оно ясно присутствует как осязаемый
объект (reg ра'i yul). И это можно сформулировать иначе: "Вначале ясно присутствует
как объект ума (уid), затем ясно присутствует как объект органов чувств (dbang ро) и,
наконец, ясно присутствует как объект тела (lus)".\\
\\
В начале следует достичь знакомства (goms ра) с проявлением божества. На этом этапе
являются божеством лишь в концептуальном или вербальном смысле (bsam ра'i blо rtog gam ngag kyi brjod pa).
Иными словам, божество ясно только с точкизрения аналитического ума,
и при этом необходимо пользоваться мыслями определенным образом (rtog dpyod kyi blo ngor).
Это называется "ясно присутствовать как объект мысли" или "ясно присутствовать как объект ума".
Когда знакомятся с проявлением божества и достигают некоторой устойчивости в этом
процессе, больше нет необходимости запечатывать медитацию (rgyas 'debs) мыслями,
используемыми на предшествующей стадии. Вместо этого божество становится какбы
видимым для полностью функционирующих чувственных способностей глаза. Не теряя из
виду эликсир безмятежности без мыслей (rtog med zhi gnas kyi rtsig), проявления
визуализации становятся отчетливыми, вплоть до зрачков глаз божества. Именно это
подразумевается под "ясным присутствием как видимого объекта" или "ясны мприсутствием
как объекта органов чувств".\\
\\
В конце этого процесса Стадия Зарождения полностью осваивается. И когда достигают
состояния зрелого видьядхары, больше не воспринимают обычные нечистые проявления,
поскольку они сливаются с мандалой божества. В этот момент актуализируется иллюзорное
тело (sgyu lus), объединенная форма божества (zung 'jug gi lha sku). Именно это
подразумевается под "ясным присутствием как осязаемого объекта" или "ясным присутствием како бъекта тела".
Наставления о четырех гвоздях, которые скрепляют жизненную силу, являются ключевыми
моментами, которые позволяют овладеть тысячечастной жизненной силой Сансары и
Нирваны. Эти сущностные наставления неразрывно связывают обычные тело, речь, ум и
активности, которые воспринимаются в Сансаре, с просветленными Телом, Речью, Умоми
Активностями, связанными с Состоянием Будды, скрепляя их жизненную силу подобно
гвоздям. Когда осваивают эти моменты, то достигают полной власти над Сансарой и
Нирваной, а также нечистые тело, речь и ум очищаются и трансформируются в
просветленные Тело, Речь и Ум. С этого самого момента осуществляют Деяния Будд (sangs
rgyas kyi mdzad pa) посредством четырех типов невообразимой просветленной активности
(phrin las) — умиротворяющей, увеличивающей, подчиняющей и гневной (zhi rgyas dbang
drag). Таким образом, четыре гвоздя, скрепляющие жизненную силу, содержат ключевые
моменты пути Стадии Зарождения.\\
\\
Сказано, что пути, лишенные этих принципов, подобны "бегу на месте", то есть такая
активность приносит лишь усталость. Правда состоит в том, что эти пути не приводят ни к
чему, кроме потери сил. Кроме того, хотя мы классифицируем божеств, на которых
медитируем, как божеств мудрости или мирских, это разделение на самом деле делается на
основе того, была или не была постигнута (rtogs) пустотность. Иными словами, пока
обладают гвоздем неизменного понимания Реальности (Дхарматы), даже визуализация себя в
форме мирского демона даст силу достичь высшего духовного достижения (dngos grub).
С другой стороны, те, кто не обладают гвоздем неизменного понимания, уверены в том, что
они и божество — разные. Такие индивидуумы не получат высшего духовного достижения,
несмотря на медитацию на запредельном божестве йидаме. Ведь они заняты тем, что
упражняют свой ум быть гневным и страстным. Очень скоро их ум становится упрямым, как
у свирепых варваров, которые увлечены ритуалами жертвоприношений и другими
бессердечными практиками. Эти люди на самом деле связаны в своей практике с могучими
мирскими духами, под чье влияние они и подпадают. И хотя они могут называть это Тайной
Мантрой, в действительности это называется Учением Демонов, от которого предостерегает
"Калачакра-тантра". В таком подходе определенно сбиваются с пути и кончают тем, что
становятся демоном или духом.
\end{siderules}

\subsection{Привнесение переживаний на путь}

Седьмая тема содержит две части: десять вещей (shes par bya ba), которые следует
понять и шестьо сновополагающих обетов самай (dam tshig). Относительно первой сказано,
что Тайная Мантра должна практиковаться подобно десяти познаваемым вещам, начиная с
иллюзиии миража. Таким образом, говорится, что Тайная Мантра должна практиковаться с
десятичастным пониманием. И это относится ко всем случаям созерцания мандалы божеств,
в медитативном или постмедитативном состояниях.\\
\\
Десятичастное понимание таково:

\begin{tabular}{ll}
 1 & садханы подобны иллюзиям;\\
 2 & все названия и слова не обладают сущностью (snying ро),\\
   & подобно миражу, который обманывает дикихжи вотных;\\
 3 & все активности подобны снам;\\
 4 & все вещи лишены истинной природы, \\
   & подобно отражениям в зеркале;\\
 5 & все места и земли подобны городам гандхарвов;\\
 6 & все звуки пусты от самобытия (ngo bo), подобно эху;\\
 7 & все чудесные формы подобны отражениям \\
   & луны в воде — проявляются, \\
   & но не имеют истинного существования;\\
 8 & различные медитативные погружения самадхи \\
   & подобны пузырям на воде;\\
 9 & все излучения и поглощения многообразны, \\
   & подобно оптическим иллюзиям;\\
10 & все чудесные проявления возникают различными способами,\\
   & но не имеют собственных характеристик (mtshan nyid), \\
   & подобно магическим иллюзиям.\\
\end{tabular}

Вторая тема связана с шестью основополагающими обетам и самаями:\\

\begin{tabular}{ll}
1 & никогда не переставать стремиться к учителю, \\
  & который дает сущностные наставления (man ngag);\\
2 & стараться создавать благоприятные условия для \\
  & медитации и отбрасывать неблагоприятные;\\
3 & не позволять угасать своему медитативному погружению, \\
  & чем бы ни занимался; \\
4 & не отказывайся от своего божества; \\
5 и 6 & сохранять суть своей медитации и \\
  & поведения в тайне от тех, \\
  & кто не является подходящим учеником.
\end{tabular}
\\
\subsection{Результаты пути, связанные с [Уровнем] \\ Четырех Видьядхар — Держателей Ведения}

\subsubsection{Приближение}

Во втором разделе объясняются уровни достижения, связанные с [уровнем] Четырех
Видьядхар, в соответствие с особым путем разделения этих четырех уровней на четыре части
Приближения и Достижения (bsnyen sgrub). Все это относится к практике Приближения, в
которой ум сосредотачивают на тонком самадхи, единой печати (phyag rgya gcig) и других
аспектах, привлекаемых для визуализации трех объектов. Такая медитация является
причиной того, что проявляются качества, связанные с Путями Накопления и Соединения.
Из взаимозависимой цепи, связанной с применением внимательности (dran ра) и прочего
тонкие энергии пяти элементов входят в центральный канал как переживание этого знания
(shes bya'i nyams su). Есть определенные знаки, которые отмечают возникновение этого
процесса, включая проявления дыма, миражей, вспышек и безоблачного неба (smig rgyu dang
srin bu me khyer dang sprin med kyi nam mkha'). Все это следует понимать как знаки того, что
начинают разворачиваться проявления медитативных переживаний.\\
\\
Кроме того есть некоторые знаки, которые появляются во сне. Неоднократно видеть
себя обнаженным — это знак того, что привычные тенденции очищаются. Восхождение по
лестнице в небо — это знак обретения Достижений (thob ра). Сидеть верхом на снежном льве
или на слоне — это знак преодоления препятствий (non ра). Есть также знаки, связанные с
получением предсказаний, например, появляющиеся образы божеств могут улыбаться. Если
коротко, то в "Ваджрной структуре (rdo rje bkod ра)" объясняется, что есть невообразимое
количество знаков, которые указывают на развитие на пути. Однако, радоваться им как
знакам своего превосходства, значит поддаваться влиянию демонов. Поэтому следует делать
все, что бы оставаться свободным отн адежд на что-то позитивное.\\
\\
И точно также, когда медитируют на стадии внешнего жара (phyi'i drod), то видят
мельчайшие частицы, буквы, символы, атрибуты рук, тонкие формы (rdul phran dang yi ge
dang phyag mtshan dang sku phra mo) и другие знаки. Воспринимаются также цвета и объекты
(kha dog dang skye mched), связанные с огнем, водой, ветром и другими подобными
факторами. На стадии внутреннего жара, в результате медитации на божество, движение
дыхания (dbugs) становится неощутимым. На стадии тайного жара все объекты (spyod yul)
постигаются как иллюзорные. Без необходимости согласованных усилийи без опоры на
причины и условия, все явления (chos thams cad) постигаются как просветленные внутри
базового Пространства (Пространства Основы?) (dbyings). В результате этого все
переживания ясно проявляются как Мудрость.\\
\\
Естьм ножество знаков, которые возникают в этом процессе медитации. Внешними
знаками являются проявления светов, звуков и запахов. Формы божеств могут смеяться,
масляные лампы могут сами собой загораться, а капалы могут взлетать в воздух. Кроме того,
можно наслаждаться хорошим самочувствием и ощущением восторга. Внутренние знаки
состоят в увеличении сострадания, ослаблении привязанности, беспристрастном отношении,
большей внимательности по отношению к обетам самаи, а также в добром расположении по
отношению к Гуру и кдрузьям по Дхарме. Также могут ослабевать надежды на позитивный
результат, привязанность к Сансаре и страх демонов. Как и прежде следует отбросить
ощущение радости от того, что получаешь нечто подобное.\\
\\
В результате достижения вершины (rtse mo) появляются различные знаки, которые
отмечают смешение ума и проявлений. Например, пять омрачений-клеш больше не
возникают в связи с внешними объектами, а также пять внешних элементов больше не могут
повредить телу.\\
\\
На стадии терпения (bzod ра) вс епроявления становятся податливыми (мягкими), а
также возникают некоторые знаки того, что появляется контроль над умом и проявлениями.
Например, из песка тогда может появляться золото, а также вода из сухой земли и деревья из
древесного угля. При этом тело не обязательно меняет свое обычное состояние, но ум
созревает (smin ра) в форму божества.\\
\\
Эта стадия называется «Полностью созревшим Видьядхарой» (видъядхарой созревания)
(rnam par smin pa'i rigs 'dzin)". В "Обширной иллюзии (sgyu' phrul rgyas pa)"
сказано:

\begin{verse}
Подобно сургучу и рельефу на печати,\\
Великая Печать, перед самым своим достижением,\\
Является нечем иным как совершенной и могущественной формой.
\end{verse}

В этом отрывке "рельеф на печати (rgya mig)" является метафорой тела (lus), тогда как
"сургуч" относится к уму. Смысл этого связан с осуществлением Великой Печати
(махамудры). И в этом отрывке указывается, что если умирают на этой стадии, недостигнув
высшего состояния (chos mchog), то Махамудра достигается в промежуточном состоянии-бардо,
подобно тому как глиняная фигурка (sa' tsa tsha) выходит из формы. Причина этого
состоит в том, что тело, которое рождается (родилось? пр.вр.?) в результате созревания
кармы, будет отброшено, а ум преобразится в форму божества. В "Этапах пути" сказано:

\begin{verse}
Когда Приближение практикуют в течение шести месяцев, \\
Ваджрное тело пока еще не достигается.\\
В результате небольших усилий, а также некоторых условий и благопожеланий,\\
Остаточное тело, обусловленное концепциями, остается.\\
Благодаря осознаванию следуют (идут?) к состоянию Ваджрадхары.
\end{verse}

С другой стороны, когда достигается мощное состояние самадхи, связанное с уровнем
высшего состояния, получают полный контроль над жизненными процессами.В тантре сказано:

\begin{verse}
Когда достигается состояние видьядхары,\\
Наделенное силой долгой жизни,\\
То обретают контроль и достигают высшего состояния\\
В течение шести, двенадцати или шестнадцати месяцев.
\end{verse}

\subsubsection{Близкое Приближение}

Говорится, что путь видения (mthong ba'i lam) осуществляется во время практики
Близкого Приближения (nye bar bsnyen ра). В тантре сказано:

\begin{verse}
Совершенство подношения, включающее все окружение,\\
Прилагай усилия, как в традиции (стадии) Приближения.
\end{verse}

На этой стадии, независимо от того как медитируют, от единственного состояния
великого таинства (gsang chen rigs gcig) и до сконструированной печати (phyag rgya spros
ра), ясность и устойчивость визуализации смешивается с Реальностью, природой великого
блаженства (chos nyid bde ba chen po'i don). Такая практика приводит к тому, что на самом
деле проявляются качества семи факторов просветления (byang chub yan lag bdun gyi yon tan).
Когда ум достигает пути Видения, возникают пять форм ясновидения (mngon par shes ра),
четыре чудесные эманации (cho 'phrul) и другие способности, связанные с этой стадией.
Кроме того появляется способность слышать Учение-Дхарму непосредственно от Будд
Нирманакаи. Далее, когда становится явной истинная природа объектов (don rang gi mtshan
nyid mngon du byas pa), великое и высшее состояние доводится до своего совершенства
(mthar phyin), а загрязненное тело превращается в Ваджрное Тело. И поскольку эта форма
(стадия?) свободна от рождения и смерти, она называется "Видьядхарой контролирующим
продолжительность жизни" (tshe la dbang ba'i rigs 'dzin). В "Этапах пути" сказано:

\begin{verse}
Вместе с обретением достижения восьми собраний,\\
Природу видят, в нее входят и достигают совершенства.\\
Поэтому загрязнения тел, миров\\
И мест рождения приходят к концу.\\
Становясь Ваджрным Телом, \\
семьей жизни (tshe yi rigs),\\
Все, что видят, есть Нирвана.\\
Без отбрасывания тела достигается \\
уровень Состояния Будды.\\
Все страхи исчезают, а чудесные эманации \\
становятся совершенными.
\end{verse}

Данная стадия осуществления (sgrub) подобна той, что великий ачарья
Падмасамбхава достиг в пещере Маратика, где Мандарава была его партнершей по практике,
и где он достиг состояния, свободного отрождения и смерти, посредством практики
непосредственной причины (nye rgyu'i spyod ра).

\subsection{Достижение}

Практика Достижения (sgrub ра) описывается как совершенствующая путь медитации
(sgom pa'i lam). Относительно этой темы в тантре сказано:

\begin{verse}
Посредством достижения (осуществления) усердствуй на пути медитации,\\
И тогда достигнешь божественного Радужного Тела Махамудры.
\end{verse}

Что касается характеристик, то природа этого медитативного состояния свободна от
заблуждений ('khrul ba) и поэтому подобна Состоянию Будды. Однако между ними есть
разница в связи с постмедитативным состоянием (rje sthob). На этой стадии энергия,
связанная с равенством Реальности (chos nyid mnyam par rtsal) все еще нуждается в
очищении. На основе мандалы тела различные чудесные эманации достигаются в состоянии
недвижимого медитативного сосредоточения (mi g.yo ba'i ting nge 'dzin). В результате этого
природа мандалы тела проявляется в радужной форме, а ментальные загрязнения, связанные
с девятью уровнями-бхуми (sa dgu), очищаются. Далее, восьмичастный благородный путь
полностью совершенствуется в форме Ясного Света Мудрости (ye shes 'od gsal), свободного
от тонких характеристик (mtshan mа). Благодаря этому могут получать как ментальные, так и
символические учения от Самбхогакаи. В тантре сказано:

\begin{verse}
Наше тело становится печатью победоносных,\\
И божество становится явным посредством медитации.\\
Украшенное главными и вторичными Признаками,\\
Оно является Видьядхарой Махамудры — Великой Печати.
\end{verse}

В этой категории есть различные подразделы. Находящиеся на уровнях со второго по
пятый известны как "Видьядхары ваджры". Поскольку их постижениеявл яется ваджрным,
оно уничтожает загрязнения каждого отдельного уровня. Находящиеся на шестомуровне
сосредотачиваются на практике Совершенства мудрости (shes rab kyi pha rol tu phyin pa) и
поворачивают колесо Дхармы. Именно поэтому они называются "Видьядхарами Колеса
[Учения]". Находящиеся на седьмом уровне называются также, поскольку благодаря своим
искусным средствам они умело работают подобно колесу. Находящиеся на восьмом уровне
обладают неконцептуальной Мудростью (mi rtog pa'i ye shes), подобной драгоценности, и
достигают контроля над своим совершенно чисты мпотенциалом (rang gi khams). Поэтому
они называются "Видьядхарами драгоценности". Находящиеся на девятом уровне
называются "Видьядхарами лотоса", поскольку они свободны от привязанности и могут
создавать чистые миры (zhing sbyong), а также работать на благо других. Находящиеся на
десятом уровне делают совершенной свою просветленную активность, благодаря которой
они приносят благо всем живым существам. Именно поэтому их называют "Видьядхарами
меча". В "Этапах пути" сказано:

\begin{verse}
Вторая Самбхогакая, семья печати,\\
Состоит из видьядхар\\
Ваджры, колеса, драгоценности, лотоса и меча.
\end{verse}

Примером этого уровня достижения является великий ачарья Падмасамбхава, когда
он достиг реализации в Янглешо, что в Непале. Там он, опираясь на мандалу Славного
Вишуддха Херуки, продемонстрировал то, как достичь уровня Видьядхары Махамудры —
Великой Печати. И это также то место, где он достиг четвертого типа просветленной
активности.

\subsection{Великое достижение}

Четвертое — это практика Великого Достижения (sgrub ра chen ро), которая
описывается как Стадия не-учения (mi slob pa'i lam). В связи с этим сказано:

\begin{verse}
Когда знаки становятся устойчивыми,\\
Демонстрируй Великое Достижение.
\end{verse}

На этом этапе предшествующие пути были полностью пройдены, и упор делается на
групповой практике, которая позволяет собранию, вместе с сотней тысяч, объедениться с
просветленным состоянием. В этой мандалете, кто относятся к спонтанной семье (lhun gyis
grub pa'i rigs), побольшей части равны Буддам относительно высших качеств, которыми они
обладают. Тем не менее, в традиции Мантры имеется мгновенное развитие, которое
происходит в ходе индивидуальных фаз (bye brag phyed ра) этой стадии, в результате чего
процесс тренировки доводится до совершенства (mthar thug). И когда это происходит, то
достигается Состояние Будды без тренировки, а также непосредственно встречаются с
Дхармакаей. Иными словами, на вершине этого состояния достигают отбрасывания и
постижения (spangs rtog). Именно так победоносный Падмасамбхава описывает [уровень]
"Видьядхары Спонтанного Присутствия (lhun grub rigs 'dzin)". В "Этапах пути" сказано:

\begin{verse}
Врезультате совершенствования силы предшествующих семей,\\
Как уже было объяснено, нечистое очищается,\\
А тройное знание Состояния Будды развертывается.\\
Так достигается спонтанно присутствующая семья.
\end{verse}

Может возникнуть мысль, что считать спонтанную семью относящейся
исключительно к уровню Состояния Будды не вполне правомерно. Однако, это не так.
Основное намерение использовать четыре семьи видьядхар, которое обнаруживается в
традиции мантр, состоит впредставлении пяти путей традиции сутр. И это подтверждается
общим смыслом того, что внутренний смысл тантр объясняется посредством шести пределов
и четырех принципов (mtha' drug dang tshul bzhi). Таким образом, Великое Достижение, как
это объясняется в контексте продвижения через четыре раздела Приближения и Достижения,
также считается целью практики, в которой нечего устранять или достигать. Факт состоит в
том, что этот подразумеваемый смысл устраняет любое подобное противоречие.\\
\\
Этот спонтанный видьядхара классифицируется как уровень состояния Будды. Тем не
менее есть дальнейшие уровни, которые достигаются, когда природа этой стадии достигает
полной силы. Высший уровень, подобный лотосу безпривязанности, возникает, когда
пребывают не загрязненными никакими негативными изъянами, независимо от того как
безотносительное знание используется при анализе. Это также относится к уровню Великого
накопления колеса - чакры слогов (yi ge 'khor lo tshogs) и различных неописуемых
просветленных качеств, таких как тридцать два возвышенных главных признака и
восемьдесят второстепенных. И , наконец, также есть тринадцатый уровень Ваджрадхары, на
котором радуются неистощимому колесу украшений (mi zad pa'i 'khor lo), просветленным
Телу, Речи и Уму Татхагаты. В "Обильном собрании украшений (rgyan stug ро bkod ра)"
сказано:

\begin{verse}
Несмотря на то, что они пробуждены\\
В высшей сфере Акаништхи,\\
Совершенные Будды не осуществляют\\
Свою просветленную деятельность в сфере желания.
\end{verse}

Как говорилось прежде, индивидуальные уровни традиции мантры пересекаются в
одном моменте переживания (spyod yul skad cig ma), благодаря которому Просветление
воплощается в Акаништхе и осуществляются бесчисленные просветленные активности,
такие как двенадцать действий беспрепятственных искусных средств (thabs mа' gags pa'i
mdzad ра). Эти классификации делались в свете причинных связей, которые создавались
автоматически (rang gir byed ра) и объяснялись как есть на самом деле. В тантре сказано:

\begin{verse}
Когда достигают вершины, спонтанного осуществления,\\
то действуют как регент-\\
Подлинно поворачивая тайное колесо,\\
Уча подходящей Дхарме всех,\\
А также демонстрируя двенадцать активностей (Деяний).
\end{verse}

Примером этого уровня достижения является великий ачарья Падмасамбхава в его
нынешнем состоянии, в котором он будет пребывать до скончания Сансары. Пребывая на
уровне Видьядхары спонтанного присутствия, Падмасамбхава не отделен от всех Будд. Он
учит Дхарме свое чистое окружение в сфере Акаништха Лотосового света и никогда не
устает проявлять сострадательную активность ради блага живых существ.\\
\\
Вместе с этим сам великий мастер наделяет видьядхару силой контролировать
продолжительность жизни, а также Великой печатью. Таким образом, это не описывалось
большинством из его последователей. Однако то, что я объяснил здесь, соответствует моему
собственному постижению, которым я обязан второму Будде (Падмасамбхаве), чьи лучи
сострадания заставили распуститься лотос моего ума. \\
\\
Некоторые считают, что Махамудра-Великая Печать охватывает с первый по
седьмой уровни-бхуми (sa), тогда как состояние спонтанного присутствия (lhun grub)
относится к трем чистым уровням. Однако это не так. И это подтверждается только что
процитированной цитатой из "Этапов пути". В связи с этим Всеведущий писал:
"Утверждение некоторых ученых, что состояние Махамудры Великой Печати охватывает с
первого по седьмой уровни, и что состояние спонтанного присутствия относится к трем
чистым уровням, соответствует неправильному пониманию. Причина этого состоит в том,
что развитие четырех видов Видьядхар охватывает всю стадию начинающего вплоть до
Состояния Будды". \\
\\
Индивидуумы, которые в основном уже собрали два накопления и достигли великой
силы относительно их знания и медитативного погружения, могут проходить через эти пути
более непосредственно. Такие индивидуумы мгновенно переходят с Пути Соединения на
Пути Видения и Медитации. Благодаря этому, они следуют совершенному пути (mthar phyin
pa'i lam). Также есть некоторые, с исключительными способностями, которые переходят
прямо с Пути Соединения к Состоянию Будды. В тантре сказано:

\begin{verse}
Некоторые совершенствуют пять истинных Тел в шестнадцать\\
От самого состояния освоения (dbang sgyur rigs),\\
Тогда как другие развиваются от состояния Махамудры - Великой Печати\\
До непревзойденного состояния Самантабхадры.
\end{verse}

Ввиду этого, может возникнуть вопрос, не достигал ли второй Будда-Ачарья
Падмасамбхава, реализации постепенно, как отмечается в примерах. Однако, это не так,
поскольку в таких текстах как "Просветление Вайрочаны (rnam snang mngon byang)" и
"Таинства капли луны (zla gsang thig lе)" указывается:

\begin{verse}
В блаженной сфере, известной как Акаништха,\\
Будды становятся полностью просветленными,\\
А затем излучают Просветление здесь.
\end{verse}

Так было и с Буддой Шакьямуни. Хотя он стал просветленным и совершенным,
достигнув отбрасывания и постижения неисчислимые кальпы тому назад, тем не менее, он
продемонстрировал Двенадцать Деяний (mdzad ра bcu gnyis) в этом мире.\\

\begin{siderules}
Во втором разделе объясняется, как следуют путям и уровням видьядхры, а также
обсуждаются четыре раздела Приближения и Достижения. Вобщем, эти четыре практики
могут применяться в различных контекстах. Они могут применяться в контексте одного
ритуала, или же с временами и количествами, связанными с практикой. Любой вариант
приемлем. Однако в данном контексте они описываются в связи со стадиями и путями.
На пути накопления знакомятся и обращаются к божеству Посвящения. Затем, не делая
ничего, что противоречит обетам самаи, которые являются жизненной силой Посвящения,
божество-йидам рассматривается как неотделимый от нашего ума. Когда четыре практики
Пути Накопления завершены, усиливают практики, соответствующие Пути Соединения.
Приступив к таким практикам актуализируют знаки, которые отмечают стадии жара,
вершины и принятия (drod rtse bzod). Далее, также актуализируется великое высшее
состояние (chos mchog chen ро), «Видьядхары полного созревания» (rnam smin rig 'dzin).
Таково Приближение.\\
\\
На Пути Видения подлинный Ясный свет возникает непосредственно в нашем потоке ума.
Затем огонь мудрости очищает элементы и нечистое тело. В этот момент еще больше
приближаются к божеству йидаму, обнаружив состояние, которое выходит за пределы
рождения и смерти. Таково Близкое Приближение.\\
\\
Далее, на Пути Медитации актуализируют состояние Видьядхары Махамудры, которое по
форме тождественно божеству йидаму. Таково достижение, означающее, что наше
собственное благо и благо других осуществляются одновременно.\\
\\
На пути Освобождения усилия, которые прилагают ради собственного блага, прекращаются.
В постмедитативном состоянии (rjes thob) на этой стадии, просветленные активности,
которые осуществляют на благо других, не отличаются от активностей Татхагат. Таково
Великое Достижение.\\
\\
В контексте пути накопления, тонкие энергии пяти элементов входят в центральный канал,
порождая пять обычных знаков. Когда ветер земли входит в центральный канал — появляются
такие знаки как дым (du ba); когда входит ветер воды -- появляются миражи (smig sgyu); когда
входит ветер огня — появляются искры (mе khyer); когда входит ветер ветра — появляются
огоньки (mar me); когда входит ветер пространства -- появляется пространство. Эти пять
являются знаками, которые проявляются непосредственно для органов чувств. В этот момент
все, на что смотрят, проявляется как расплывчатое и туманное, подобно дыму; мерцающее,
подобно миражу; вспыхивающее, подобно искрам; оранжевое (dmar ser), подобно
светильнику или жеподобно безоблачному небу. Это происходит только в главном
медитативном равновесии.\\
\\
Описание этих стадий жара, вершины и принятия на пути соединения легко понять. Однако,
в описании стадии высшего состояния, которое связывается с Видьядхарой полного
созревания, есть трудные места, которые начинаются со слов: "Подобно сургучу и рельефу
печати". Слов "рельеф" здесь соотносится с вырезанным рисунком напечати, тогда как
"сургуч" - это материал, который и спользуется для нанесения печати. Однако, во всех
смыслах и значениях, этот пример схож с примером извлечения глиняной фигурки из
формы. Смысл этого состоит в том, что хотя ум может и созревает в форму божества, тем не
менее все еще нет выхода за пределы физического тела, которое является созреванием
прошлой кармы.\\
\\
Далее следует фраза: "Высшее состояние контроля (dbang bsgyur rigs kyi dam pa)". И это
относится к тому факту, что на этой стадии достигают Тела света (dwangs mа 'od kyi lus) как
опору, а также становятся равными благой участью (skal ba) богам сферы формы. В этом
контексте оно похоже на Всемогущего царя (dbang phyugs), или Владыку этих чистых небес.
Точно также можно описывать совершенно Просветленное божество-йидама как "Всемогущего".
В случае «Видьядхары, обладающего контролем над продолжительностью жизни», причина
определяется как знание. Это относится к воззрению, рассматривающему нераздельную
природу Сансары и Нирваны. И условие здесь определяется как вхождение, то есть самадхи.
Следовательно тот, кто сделал совершенными четыре практики пути накопления, может
достичь пути видения, используя различные усиливающие практики, которые
подразумевают группу практик на пути соединения. \\
\\
В этот момент в данном процессе истощаются три загрязнения (zag ра) и достигают
нерожденного и неумирающего Ваджрного тела. Три загрязнения - это загрязнения тела (lus),
сферы (khams) и места рождения (skye gnas). Первое предполагает невозможность
контролировать процесс смерти и болезни. Второе связано с невозможностью использовать
состояния медитативного сосредоточения, чтобы пожеланию рождаться в сфере желания. И
третье предполагает невозможность контролировать процесс рождения одним из четырех
способов.\\
\\
Четыре чуда (cho 'phrul) - это чудеса вербального выражения (kun brjod), подходящих учений
(rjes bstan), магических проявлений (rdzu 'phrul) и манифестации явлений (chos snang). И
хотя в "Сжатом понимании (dgongs 'dus)" представлены лишь два различных вида, эту
категоризацию очень легко объяснить. Чудо вербального выражения связано с
просветленным умом и вданном случае соотносится сознанием умов других существ,
благодаря собственным психическим силам. Чудесные проявления могут изменять
восприятие других существ, например, внушая веру тем, у кого ее нет. Чудо подходящих
учений относится к просветленной речи, в связи с учением любой из трех Колесниц Дхармы,
которое более подходит для покорения учеников. И, наконец, чудо манифестации явлений
предполагает способность осуществлять такие вещи как землетрясение в шести
направлениях, а также манифестация восемнадцати великих знамений (ltas chen ро) вовремя
учения Дхармы.\\
\\
Путь медитации связан с видьядхарой Великой печати. В данном контексте появляется
фраза: "завесы, связанные с девятью уровнями". Это относится к девяти обычным уровням
на пути медитации, со второго уровня и выше.\\
\\
Спонтанно присутствующий видьядхара связан с путем Освобождения, где мы находим
фразу: "тройная мудрость Состояния Будды". Это является синонимом трем типам
осознавания, которым учат в "Сутре обширной игры (rgya che rol ра)", где сказано, что наш
Учитель достиг трех типов осознавания, когда он осуществил (mngon du mdzad ра)
Просветление. Первое — это знание прошлых жизней на протяжение невообразимого
промежутка времени. Второй тип — это знание смерти и рождений, то есть знание
обстоятельств смерти и будущих рождений каждого существа. Следовательно, обладают
неописуемым знанием смерти и рождения. И, наконец, есть знание истощения загрязнений.
Это точное знание того, истощаться или незагрязнения, как относительно себя самого, так и
относительно других.\\
\\
Шраваки, пратьека Будды и бодхисаттвы обладают лишь подобием этих трех видов знания.
Тогда как у совершенного Будды эти три вида знания достигают высшего уровня и являются
совершенными. Таким образом они беспримерны.\\
\\
Описание особых знаков и примет (mtshan dpe) можно найти в тантрических комментариях.
В естественно проявляющейся сфере акаништхи различные сферы, чудесные дворцы,
центральные фигуры, свиты и т.д., возникают как чудесная игра единой мудрости (ye shes
gcig gir nam rol). В этом контексте шестнадцать существ второстепенных классов и их
супруг и образуют свиту. Таковы знаки. Присутствие пяти божеств на короне (dbu rgyan)
каждого мужского божества представляет приметы. Все вместе - это внутренние знаки и
приметы.
\end{siderules}
